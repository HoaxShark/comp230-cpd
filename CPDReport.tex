% Please do not change the document class
\documentclass{scrartcl}

% Please do not change these packages
\usepackage[hidelinks]{hyperref}
\usepackage[none]{hyphenat}
\usepackage{setspace}
\doublespace

% You may add additional packages here
\usepackage{amsmath}

% Please include a clear, concise, and descriptive title
\title{Personal Reflection}

% Please do not change the subtitle
\subtitle{COMP230 - CPD Report}

% Please put your student number in the author field
\author{1706966}

\begin{document}

\maketitle

\section{Introduction}
It is that time of year again where I need to complete this continuing professional development report. I have found these to be some of the trickier reports we must compile, as they require that I look closely at areas in which I need to improve. Although looking for these ways to improve is something I do continuously throughout the year, writing it down and formulating explanations to my choices is challenging. I believe this is due to my own uncertainty of what is required to be seen in these reports, along with the general discomfort caused by reflecting deeply about oneself.

I will endeavour to look for five key skills where I have noticed a need for improvement during the last term. These skills will each fit into one of five main domains; affective, interpersonal, dispositional, cognitive and procedural. I will justify why these skills have relevance and importance for me and my career goals, contemplate why I am not succeeding with these skills as well as I would like to, and how this affects the quality of my work. Concluding with some measurable goals that will help me overcome these failing key skills.

\section{Interpersonal}
In the interpersonal domain I would like to look deeper at my receptiveness to feedback. Specifically how I instantly respond to negative, but well-intentioned feedback. I can only think of one instance where this happened but I annoyed myself with it enough that it has stuck with me. I believe that not considering this continuously may allow for similar experiences to occur. When I receive negative feedback I have an insatiable urge to quickly explain my side. Although this comes with the best of intentions, wanting to make the other person see the reasons for my choices, I believe it likely comes off as being very defensive and possibly a bit argumentative. 

It is unlikely that any career I go into won't require me to work with others and receive feedback over the years, therefore it is important to work on this skill to ensure I work effectively and can learn well from my colleagues.
\subsection{Goal}
Until the end of this academic year, if I find myself in a situation where I do not fully understand a piece of feedback given, I will question it without using defensive language.

\section{Cognitive}
Although programming patterns are on my radar and I have often thought that I need to look more closely at them, it has been a failure on my part not looking into these as much as I feel is necessary. I believe having a knowledge of these patterns will give me effective and efficient solutions to implement often throughout my future code base. I expect that not having a decent grasp on these patterns will look bad for me when searching for a job in the industry.
\subsection{Goal}
Every week I will read one section of gameprogrammingpatterns.com, this will complete within twenty weeks.

\section{Procedural}
It has come to my attention that my ability to debug problems in code is lacking. I utilise the knowledge of my colleagues and staff when I am presented with a problem, and see how well they navigate through the IDE, often presenting new ways of finding information that I was unaware of. Having a better understanding of available keyboard shortcuts can help save me time with debugging issues and even general workflow. 

Another key area on which I would like to improve on is what the different sections of the IDE can do for me. I have only learnt how to use these types of software by seeing it used, which has given me a basic working knowledge but there is a lot more that can be achieved were I to have a deeper understanding of the tools available. Although in the future I will most certainly be using different IDE's, the process of learning about them and what they can do will likely be similar. I believe it will be good practice for myself to start now and research each IDE that I use.
\subsection{Goal}
For each new IDE or programming language I start using, I will spend 3 hours of my time looking at guides on how to use them efficiently.

\section{Affective}
This section looks at how I have seen positive feedback to play a large part in invigorating my peers to create fantastic work. Hopefully everyone knows how being given praise for your hard efforts feels warming and tends to leave you with a drive to want more.

I have made attempts to provide my peers with this, in an effort to spur them on with their work, but I have seen myself dwindle down from the amount it happens. I great opportunity to do this which I have yet to capitalise on is in the PO meetings with the anonymous feedback. Making my peers feel accomplished with there work may not help my work specifically but when working in a team environment doing something for the greater good of the team should have worth in itself. Being able to do this well will hopefully provide better end products throughout the terms and in a team job environment.
\subsection{Goal}
For each anonymous PO reviews for the coming term, I will give everyone some positive feedback where applicable.

\section{Dispositional}
My issue here slightly harks back to the dispositional section of my last CPD.  In this I claimed that I need to work on my projects earlier and save myself from rushing towards the end of a deadline. I have managed fairly well at this. Balancing my work throughout the year has presented the new challenge of running the risk that I become overly focused and forget how projects relate to assignment briefs.

I'm very aware of how poor my memory can be, so I should really be working harder to combat flaws like this. Making sure I always have an idea of what all my assignments require will allow me to correctly allocate time to each section across the term, hopefully meaning that I work smarter and no sections get missed until close to their deadlines. This ability to remind myself and gain a constant knowledge of the jobs that need to be completed, will hopefully flow fantastically into a working environment and even improve my ability to organise my personal life, once it is a natural routine for me.
\subsection{Goal}
At the beginning of each month I will read through all the assignments I have for the term and make a note of anything vital for that month, along with jobs that would work well given my other tasks for the coming month.

\section{Conclusion}
I am happy with the goals I have set out in this CPD. Everything selected not only has its part to play in improving my work at university but will also set me up to be a much more employable person overall. Having recognised how I have achieved previously outlined CPD goals this year, I look forward to achieving these ones, and seeing what I may discover about myself in this coming term. 

\end{document}
