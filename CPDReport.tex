% Please do not change the document class
\documentclass{scrartcl}

% Please do not change these packages
\usepackage[hidelinks]{hyperref}
\usepackage[none]{hyphenat}
\usepackage{setspace}
\doublespace

% You may add additional packages here
\usepackage{amsmath}

% Please include a clear, concise, and descriptive title
\title{Personal Reflection}

% Please do not change the subtitle
\subtitle{COMP230 - CPD Report}

% Please put your student number in the author field
\author{1706966}

\begin{document}

\maketitle

\section{Introduction}
It is that time of year again where I need to complete this continuing professional development report. I have found these to be some of the trickier reports we must compile, as they require that I look closely at areas in which I need to improve. Although looking for these ways to improve is something I do continuously throughout the year, writing it down and formulating explanations to my choices is challenging. I believe this is due to my own uncertainty of what is required to be seen in these reports, along with the general discomfort caused by reflecting deeply about oneself.
I will endeavour to look for five key skills that I have noticed a need for improvement in during the last term. These skills will each fit into one of five main domains; affective, interpersonal, dispositional, cognitive and procedural. Justify why these skills have relevance and importance for me and my career goals, contemplate why I am not succeeding with these skills as well as I would like to and how this affects the quality of my work. Finally setting myself measurable goals that will help me overcome these failing key skills.

\section{Interpersonal}
In the interpersonal domain I would like to look deeper at my receptiveness to feedback, more specifically how I instantly respond to negative, but well-intentioned feedback. I can only think of one instance where this happened but I annoyed myself with it enough that it has stuck with me, I believe that not considering this continuously may allow for similar experiences to occur. When I receive negative feedback I have an insatiable urge to quickly explain my side. Although this comes with the best of intentions, wanting to make the other person see the reasons for my choices, I believe it likely comes off as being very defensive and possibly a bit argumentative. 
It is unlikely that any career I go into won't require me to work with others and receive feedback over the years, therefore it is important to work on this skill as I would not like to be seen in a negative light by my co-workers, I also want them to all be very comfortable with providing me feedback as it will only help improve my work output and relationships.
\subsection{Goal}
Until the end of this academic year I will not respond to feedback other than to say thanks or if explicitly asked for an explanation.

\section{Cognitive}
My understanding of code and syntax is slowly improving, but an area in which I am currently weak on is knowing when best to use references or pointers. I need to improve my understanding of the differences between them and how best to use them. This is an important part of being able to write successful and efficient code, which will be important for me when looking for a career in software development. I have come up with a SMART goal to help push myself to learn more about this skill.
\subsection{Goal}
I will dedicate 2 hours a week for 12 weeks to researching pointers and referencing. I will spend some of that time practising the use of them by writing code snippets. This will allow me to write better, more efficient code in the future which is important if wanting to work in the software industry.


\section{Procedural}
As my understanding of how code is written has improved I have found myself wanting to know more efficient ways of solving problems. The way I currently work out most logic problems solve the problem, but i'm sure there are more efficient ways of doing so. Being able to write code that runs efficiently is a very important aspect in software development and having this skill will greatly improve my employability with companies. Therefore I have set a SMART goal to help focus on improving this skill.
\subsection{Goal}
I will spend 1 hour a day, 5 days a week for 12 weeks working on code wars challenges on-line. I will make sure to look at the most efficient solutions from other participants after completing a challenge. This will give me a chance to work out problems and then see some of the best ways it has been done, giving me a good insight into efficient logic problem solving.


\section{Affective}
During this group project I found that the lack of interest and dedication shown by the rest of the team for the project ended up rubbing off on me as well. This not only let my personal input to the project slip below what I consider to be satisfactory. This annoyed me and would not be acceptable working practice in a paid environment. As for the other skills I have a SMART goal to help avoid this issue in the future.
\subsection{Goal}
Over the course of my next group project I will work for at least 20 hours on the project over the course of the week. Having this goal will mean no matter the dedication of the team I am still putting in a satisfactory effort in to the group project.


\section{Dispositional}
I have found myself more often than not leaving work towards the end of a deadline and picking it up to do just before, although I can work in this manner I don't believe it allows me to get the best results from my work. With some projects I would have liked to have had time to change some smaller aspects of the work and had I started sooner I would have had the time. When working for a company it will not be acceptable to be lax with your time and projects will be expected to be worked on from start to finish date. 
\subsection{Goal}
I will take a look at assignment briefs for work when assigned and use a calendar system to organise my work time efficiently from start to end of the deadline. This will help with my personal organisation and hopefully allow me to produce better end products.

\section{Conclusion}
In conclusion, the skills that I have picked to improve on will help later when working in the industry but also help me through the summer break and throughout my remaining 2 years at university. I believe the SMART goals I have set are reasonable and will help facilitate my learning and optimise my time at university. I look forward to seeing myself improve because of them.


\end{document}
