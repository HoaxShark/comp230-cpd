% Please do not change the document class
\documentclass{scrartcl}

% Please do not change these packages
\usepackage[hidelinks]{hyperref}
\usepackage[none]{hyphenat}
\usepackage{setspace}
\doublespace

% You may add additional packages here
\usepackage{amsmath}

% Please include a clear, concise, and descriptive title
\title{Personal Reflection}

% Please do not change the subtitle
\subtitle{COMP230 - CPD Report}

% Please put your student number in the author field
\author{1706966}

\begin{document}

\maketitle

\section{Introduction}
It is that time of year again where I need to complete this continuing professional development report. I have found these to be some of the trickier reports we must compile, as they require that I look closely at areas in which I need to improve. Although looking for these ways to improve is something I do continuously throughout the year, writing it down and formulating explanations to my choices is challenging. I believe this is due to my own uncertainty of what is required to be seen in these reports, along with the general discomfort caused by reflecting deeply about oneself.

I will endeavour to look for five key skills that I have noticed a need for improvement in during the last term. These skills will each fit into one of five main domains; affective, interpersonal, dispositional, cognitive and procedural. Justify why these skills have relevance and importance for me and my career goals, contemplate why I am not succeeding with these skills as well as I would like to and how this affects the quality of my work. Finally setting myself measurable goals that will help me overcome these failing key skills.

\section{Interpersonal}
In the interpersonal domain I would like to look deeper at my receptiveness to feedback, more specifically how I instantly respond to negative, but well-intentioned feedback. I can only think of one instance where this happened but I annoyed myself with it enough that it has stuck with me, I believe that not considering this continuously may allow for similar experiences to occur. When I receive negative feedback I have an insatiable urge to quickly explain my side. Although this comes with the best of intentions, wanting to make the other person see the reasons for my choices, I believe it likely comes off as being very defensive and possibly a bit argumentative. 

It is unlikely that any career I go into won't require me to work with others and receive feedback over the years, therefore it is important to work on this skill as I would not like to be seen in a negative light by my co-workers, I also want them to all be very comfortable with providing me feedback as it will only help improve my work output and relationships.
\subsection{Goal}
Until the end of this academic year I will not respond to feedback other than to say thanks or if explicitly asked for an explanation.

\section{Cognitive}
Although programming patterns are on my radar and I have often thought that I need to look more closely at them, it has been a failure on my part to actually research more into them. It is my understanding that these patterns provide a basic structure that can be used as a solution to commonly occurring problems. I believe it is best to have a knowledge of these patterns as they will give effective and efficient solutions to implement often throughout my future code base. I expect that not have a decent grasp on these patterns will look bad for me when looking for a job in the industry.
\subsection{Goal}
Every week I will read one section of gameprogrammingpatterns.com, this will complete within twenty weeks.

\section{Procedural}
It has come to my attention that my ability to debug problems in code is lacking. When I have been stuck and asked some of the other students or staff for assistance, I see how well they navigate through the IDE and often present new ways of finding information that I was unaware of. Having a better understanding of available keyboard shortcuts can help save me time with debugging issues and even general work flow. 

The real concept I would a better grasp on here though is more knowledge on what the different sections of the IDE can do for me. I have only learnt how to use these types of software by seeing it used, which has given me a basic working knowledge but there is a lot more that can be achieved were I to have a deeper understanding of the tools available. Although in the future I will most certainly be using different IDE's the process of learning about them and what they can do will likely be similar, therefore I believe it will be good practice for myself to start now and research each IDE that I use.
\subsection{Goal}
For each new IDE or programming language I start using, I will spend 3 hours of my time looking at guides on how to use them efficiently.

\section{Affective}
During this group project I found that the lack of interest and dedication shown by the rest of the team for the project ended up rubbing off on me as well. This not only let my personal input to the project slip below what I consider to be satisfactory. This annoyed me and would not be acceptable working practice in a paid environment. As for the other skills I have a SMART goal to help avoid this issue in the future.
\subsection{Goal}
Over the course of my next group project I will work for at least 20 hours on the project over the course of the week. Having this goal will mean no matter the dedication of the team I am still putting in a satisfactory effort in to the group project.

\section{Dispositional}
My issue here slightly harks back to the dispositional section of my last CPD, there I claimed that I need to work on my projects earlier and save myself from rushing towards the end of a deadline. I have managed fairly well at this, or so I believed until just last week. Although I have been working consistently throughout the year another stumbling block has arisen. It is one that I was aware of and tried to combat at the beginning of this year but my efforts weren't enough, prompting me to bring it into this CPD. The problem is remembering what each assignment brief requires and balancing the different assignments throughout the term. I have a tendency to focus deeply on one project at a time and this had been going ok, but now at the end of term I am noticing sections of assignments I could have easily completed whilst I was doing them without knowing it was required. 

I'm very aware of how poor my memory can be, so I should really be working harder to combat flaws like this. Making sure I always have an idea of what all my assignments require will allow me to correctly allocate time to each section across the term, hopefully meaning that I work smarter and no sections get missed until close to their deadlines. This ability to remind myself and gain a constant knowledge of the jobs that need to be completed, will flow fantastically into a working environment and even improve my ability to organise my personal life, once it is a natural routine for me.
\subsection{Goal}
At the beginning of each month I will read through all the assignments I have for the term and make a note of anything vital for that month, along with jobs that would work well given my other tasks for the coming month.

\section{Conclusion}
In conclusion, the skills that I have picked to improve on will help later when working in the industry but also help me through the summer break and throughout my remaining 2 years at university. I believe the SMART goals I have set are reasonable and will help facilitate my learning and optimise my time at university. I look forward to seeing myself improve because of them.


\end{document}
